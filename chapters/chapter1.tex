\chap{Lojban As We Mangle It In Lojbanistan: About This Book}
\sect{What is Lojban?}
Lojban (pronounced \q{LOZH-bahn}) is a constructed language. Previous versions of the language were called \q{Loglan} by Dr. James Cooke Brown, who founded the Loglan Project and started the development of the language in 1955. The goals for the language were first described in the open literature in the article \q{Loglan}, published in \textit{Scientific American}, June, 1960. Made well-known by that article and by occasional references in science fiction (most notably in Robert Heinlein's novel \textit{The Moon Is A Harsh Mistress} and computer publications, Loglan and Lojban have been built over four decades by dozens of workers and hundreds of supporters, led since 1987 by The Logical Language Group (who are the publishers of this book).

There are thousands of artificial languages (of which Esperanto is the best-known), but Loglan/Lojban has been engineered to make it unique in several ways. The following are the main features of Lojban:

\begin{itemize}
\item Lojban is designed to be used by people in communication with each other, and possibly in the future with computers.
\item Lojban is designed to be neutral between cultures.
\item Lojban grammar is based on the principles of predicate logic.
\item Lojban has an unambiguous yet flexible grammar.
\item Lojban has phonetic spelling, and unambiguously resolves its sounds into words.
\item Lojban is simple compared to natural languages; it is easy to learn.
\item Lojban's 1300 root words can be easily combined to form a vocabulary of millions of words.
\item Lojban is regular; the rules of the language are without exceptions.
\item Lojban attempts to remove restrictions on creative and clear thought and communication.
\item Lojban has a variety of uses, ranging from the creative to the scientific, from the theoretical to the practical.
\item Lojban has been demonstrated in translation and in original works of prose and poetry.
\end{itemize}



\sect{What is this book?}
This book is what is called a \q{reference grammar}. It attempts to expound the whole Lojban language, or at least as much of it as is understood at present. Lojban is a rich language with many features, and an attempt has been made to discover the functions of those features. The word \q{discover} is used advisedly; Lojban was not \q{invented} by any one person or committee. Often, grammatical features were introduced into the language long before their usage was fully understood. Sometimes they were introduced for one reason, only to prove more useful for other reasons not recognized at the time.

By intention, this book is complete in description but not in explanation. For every rule in the formal Lojban grammar (given in \chapref{21}), there is a bit of explanation and an example somewhere in the book, and often a great deal more than a bit. In essence, \chapref{2} gives a brief overview of the language, \chapref{21} gives the formal structure of the language, and the chapters in between put semantic flest on those formal bones. I hope that eventually more grammatical material founded on (or even correcting) the explanations in this book will become available.

Nevertheless, the publication of this book is, in one sense, the completion of a long period of language evolution. With the exception of a possible revision of the language that will not even be considered until five years from publication date, and any revisions of this book needed to correct outright errors, the language described in this book will not be changing by deliberate act of its creators any more. Instead, language change will take place in the form of new vocabulary --- Lojban does not yet have nearly the vocabulary it needs to be a fully usable language of the modern world, as \chapref{12} explains --- and through the irregular natural processes of drift and (who knows?) native-speaker evolution. (Teach your children Lojban!) You can learn the language described here with assurance that (unlike previous versions of Lojban and Loglan, as well as most other artificial languages) it will not be subject to further fiddling by language-meisters.

It is probably worth mentioning that this book was written somewhat piecemeal. Each chapter began life as an explication of a specific Lojban topic; only later did these begin to clump together into a larger structure of words and ideas. Therefore, there are perhaps not as many cross-references as there should be. However, I have attempted to make the index as comprehensive as possible.

Each chapter has a descriptive title, often involving some play on words; this is an attempt to make the chapters more memorable. The title of \chapref{1} (which you are now reading), for example, is an allusion to the book \textit{English As We Speak It In Ireland}, by P. W. Joyce, which is sort of informal reference work about Hiberno-English. \q{Lojbanistan} is both an imaginary country where Lojban is the native language, and a term for the actual community of Lojban-speakers, scattered over the world. Why \q{mangle}? As yet, nobody in the real Lojbanistan speaks the language at all well, by the standards of the imaginary Lojbanistan; that is one of the circumstances this book is meant to help remedy.



\sect{What are the typographical conventions of this book?}
Each chapter is broken into numbered sections; each section contains a mixture of expository text, numbered examples, and possibly tables.

The reader will notice a certain similarity in the examples used throughout the book. One chapter after another rings the changes on the self-same sentences:
\begin{example}
mi klama le zarci \n
I go-to that-which-I-describe-as-a store. \n
I go to the store.
\end{example}

{\noindent}will become wearisomely familiar before \chapref{21} is reached. This method is deliberate; I have tried to use simple and (eventually) familiar examples wherever possible, to avoid obscuring new grammatical points with new vocabulary. Of course, this is not the method of a textbook, but this book is not a textbook (although people have learned Lojban from it and its predecessors). Rather, it is intended both for self-learning (of course, at present would-be Lojban teachers must be self-learners) and to serve as a reference in the usual sense, for looking up obscure points about the language.

It is useful to talk further about \exref{1.3.1} for what it illustrates about examples in this book. Examples usually occupy three lines. The first of these is in Lojban, the second in a word-by-word literal translation of the Lojban into English, and the third in colloquial English. The second and third lines are sometimes called the \q{literal translation} and the \q{colloquial translation} respectively. Sometimes, when clarity is not sacrificed thereby, one or both are omitted. If there is more than one Lojban sentence, it generally means that they have the same meaning.

Words are sometimes surrounded by angle brackets. In Lojban texts, these enclose optional grammatical particles that may (in the context of the particular example) be either omitted or included. In literal translations, they enclose words that are used as conventional translations of specific Lojban words, but don't have exactly the meanings or uses that the English word would suggest. In \chapref{3}, forward slashes surround phonetic representations in the International Phonetic
Alphabet.

Many of the tables, especially those placed at the head of various sections, are in three columns. The first column contains Lojban words discussed in that section; the second column contains the grammatical category (represented by an UPPER CASE Lojban word) to which the word belongs, and the third column contains a brief English gloss, not necessarily or typically a full explanation. Other tables are explained in context.

A few Lojban words are used in this book as technical terms. All of these are explained in \chapref{2}, except for a few used only in single chapters, which are explained in the introductory sections of those chapters.



\sect{Disclaimers}
It is necessary to add, alas, that the examples used in this book do not refer to any existing person, place, or institution, and that any such resemblance is entirely coincidental and unintentional, and not intended to give offense.

When definitions and place structures of gismu, and especially of lujvo, are given in this book, they may differ from those given in the English-Lojban dictionary (which, as of this writing, is not yet published). If so, the information given in the dictionary supersedes whatever is given here.



\sect{Acknowledgements and Credits}
Although the bulk of this book was written for the Logical Language Group (LLG) by John Cowan, who is represented by the occasional authorial \q{I}, certain chapters were first written by others and then heavily edited by me to fit into this book.

In particular: \chapref{2} is a fusion of originally separate documents, one by Athelstan, and one by Nora Tansky LeChevalier and Bob LeChevalier; \chapref{3} and \chapref{4} were originally written by Bob LeChevalier with contributions by Chuck Barton; \chapref{12} was originally written (in much longer form) by Nick Nicholas; the dialogue near the end of \chapref{13} was contributed by Nora Tansky LeChevalier; \chapref{15} and parts of \chapref{16} were originally by Bob LeChevalier; and the YACC grammar in \chapref{21} is the work of several hands, but is primarily by Bob LeChevalier and Jeff Taylor. The BNF grammar, which is also in \chapref{21}, was originally written by me, then rewritten by Clark Nelson, and finally touched up by me again.

The research into natural languages from which parts of \chapref{5} draw their material was performed by Ivan Derzhanski. LLG acknowledges his kind permission to use the fruits of his research.

The pictures in this book were drawn by Nora Tansky LeChevalier, except for the picture appearing in \chapref{4}, which is by Sylvia Rutiser.

I would like to thank the following people for their detailed reviews, suggestions, comments, and early detection of my embarrassing errors in Lojban, logic, English, and cross-references: Nick Nicholas, Mark Shoulson, Veijo Vilva, Colin Fine, And Rosta, Jorge Llambias, Iain Alexander, Paulo S. L. M. Barreto, Robert J. Chassell, Ivan Derzhanski, Jim Carter, Irene Gates, Bob LeChevalier, John Parks-Clifford (also known as \q{pc}), and Nora Tansky LeChevalier.

Nick Nicholas (NSN) would like to thank the following Lojbanists: Mark Shoulson, Veijo Vilva, Colin Fine, And Rosta, and Iain Alexander for their suggestions and comments; John Cowan, for his extensive comments, his exemplary trailblazing of Lojban grammar, and for solving the \q{manskapi} dilemma for NSN; Jorge Llambias, for his even more extensive comments, and for forcing NSN to think more than he was inclined to; Bob LeChevalier, for his skeptical overview of the issue, his encouragement, and for scouring all Lojban text his computer has been burdened with for lujvo; Nora Tansky LeChevalier, for writing the program converting old rafsi text to new rafsi text, and sparing NSN from embarrassing errors; and Jim Carter, for his dogged persistence in analyzing lujvo algorithmically, which inspired this research, and for first identifying the three lujvo classes.

Of course, the entire Loglan Project owes a considerable debt to James Cooke Brown as the language inventor, and also to several earlier contributors to the development of the language. Especially noteworthy are Doug Landauer, Jeff Prothero, Scott Layson, Jeff Taylor, and Bob McIvor. Final responsibility for the remaining errors and infelicities is solely mine.



\sect{Informal Bibliography}
The founding document for the Loglan Project, of which this book is one of the products, is \textit{Loglan 1: A Logical Language} by James Cooke Brown (4th ed. 1989, The Loglan Institute, Gainesville, Florida, U.S.A.) The language described therein is not Lojban, but is very close to it and may be considered an ancestral version. It is regrettably necessary to state that nothing in this book has been approved by Dr. Brown, and that the very existence of Lojban is disapproved of by him.

The logic of Lojban, such as it is, owes a good deal to the American philosopher W. v.O. Quine, especially \textit{Word and Object} (1960, M.I.T. Press). Much of Quine's philosophical writings, especially on observation sentences, reads like a literal translation from Lojban.

The theory of negation expounded in \chapref{15} is derived from a reading of Larry Horn's work \textit{The Natural History of Negation}.

Of course, neither Brown nor Quine nor Horn is in any way responsible for the uses or misuses I have made of their works.

Depending on just when you are reading this book, there may be three other books about Lojban available: a textbook, a Lojban/English dictionary, and a book containing general information about Lojban. You can probably get these books, if they have been published, from the same place where you got this book. In addition, other books not yet foreseen may also exist.



\sect{Captions to Pictures}
The following examples list the Lojban caption, with a translation, for the picture at the head of each chapter. If a chapter's picture has no caption, \q{(none)} is specified instead.
\begin{example}
\hyperref[img:1]{coi lojban.       coi rodo} \n
Greetings, O Lojban!    Greetings, all-of you
\end{example}

\begin{example}
(none)
\end{example}

\begin{example}
\hyperref[img:3]{.i .ai .i .ai .o} \n
[untranslatable]
\end{example}

\begin{example}
\hyperref[img:4]{jbobliku} \n
Lojbanic-blocks
\end{example}

\begin{example}
(none)
\end{example}

\begin{example}
\hyperref[img:6]{lei re nanmu cu bevri le re nanmu} \n
The-mass-of two men carry the two men \n
Two men (jointly) carry two men (both of them).
\end{example}

\begin{example}
\hyperref[img:7]{ma drani danfu} \n
\T	.i di'e \n
\T	.i di'u \n
\T	.i dei \n
\T	.i ri \n
\T	.i do'i \n
[What sumti] is-the-correct type-of-answer? \n
\T	The-next-sentence. \n
\T	The-previous-sentence. \n
\T	This-sentence. \n
\T	The-previous-sentence. \n
\T	An-unspecified-utterance.
\end{example}

\begin{example}
\hyperref[img:8]{ko viska re prenu poi bruna la santas.} \n
[You!] see two persons who-are brothers-of Santa.
\end{example}

\begin{example}
(none)
\end{example}

\begin{example}
\hyperref[img:10]{za'o klama} \n
[superfective] come/go \n
Something goes (or comes) for too long.
\end{example}

\begin{example}
\hyperref[img:11]{le si'o kunti} \n
The concept-of emptiness
\end{example}

\begin{example}
(none)
\end{example}

\begin{example}
\hyperref[img:13]{.oi ro'i ro'a ro'e} \n
[Pain!] [emotional] [social] [mental]
\end{example}

\begin{example}
(none)
\end{example}

\begin{example}
\hyperref[img:15]{mi na'e lumci le karce} \n
I other-than wash the car \n
I didn't wash the car.
\end{example}

\begin{example}
\hyperref[img:16]{drata mupli pe'u .djan.} \n
another example [please] John \n
Another example, John, please!
\end{example}

\begin{example}
\hyperref[img:17]{zai xanlerfu by. ly. .obu .jy by. .abu ny.} \n
[Shift] hand-letters l o j b a n \n
"Lojban" in the manual alphabet
\end{example}

\begin{example}
\hyperref[img:18]{no no} \n
0 0
\end{example}

\begin{example}
(none)
\end{example}

\begin{example}
(none)
\end{example}

\begin{example}
(none)
\end{example}



\sect{Boring Legalities}
This book is Copyright \copyright 1997 by The Logical Language Group, Inc.

Permission is granted to make and distribute verbatim copies of this book, either in electronic or in printed form, provided the copyright notice and this permission notice are preserved on all copies.

Permission is granted to copy and distribute modified versions of this book, provided that the modifications are clearly marked as such, and provided that the entire resulting derived work is distributed under the terms of a permission notice identical to this one.

Permission is granted to copy and distribute translations of this manual into another language, under the above conditions for modified versions, except that this permission notice may be stated in a translation that has been approved by the Logical Language Group, rather than in English.

The contents of \chapref{21} are in the public domain.
